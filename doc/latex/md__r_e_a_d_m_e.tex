Copyright (c) 2018, The E\+V\+EN Found Project

\subsection*{Development resources}


\begin{DoxyItemize}
\item Web\+: \href{https://evenfound.org}{\tt evenfound.\+org}
\item Mail\+: \href{mailto:dev@evenfound.org}{\tt dev@evenfound.\+org}
\item Github\+: \href{https://github.com/evenfound/even-network}{\tt https\+://github.\+com/evenfound/even-\/network}
\item I\+RC\+: \href{irc://chat.freenode.net/#evenfound-dev}{\tt \#evenfound-\/dev on Freenode}
\item Translation platform (Pootle)\+: \href{https://translate.evenfound.org}{\tt translate.\+evenfound.\+org}
\end{DoxyItemize}

\subsection*{Introduction}

E\+V\+EN Network � is an open cross-\/chain platform with E\+V\+EN cryptocurrency that allows users to interact with more than one blockchain at a time.

{\bfseries Cross-\/chain\+:} Transactions involving more than one blockchain through a single entry point.

{\bfseries Reliability\+:} A high degree of protection for digital assets due to unique cryptographic solutions.

{\bfseries Control\+:} Private keys are always under the control of a user.

{\bfseries Scalability\+:} The network will continue to run quickly even as the number of participants grows, and external services can be integrated into it using the open A\+PI.

\subsection*{About this project}

This is the core E\+V\+EN Network Platform A\+PI all base interfaces, simulation applications and prerequisites including. It is open source and completely free to use without restrictions, except for those specified in the license agreement below. There are no restrictions on anyone creating an alternative implementation of E\+V\+EN Platform that uses the protocol and network in a compatible manner.

As with many development projects, the repository on Github is considered to be the \char`\"{}staging\char`\"{} area for the latest changes. Before changes are merged into that branch on the main repository, they are tested by individual developers in their own branches, submitted as a pull request, and then subsequently tested by contributors who focus on testing and code reviews. That having been said, the repository should be carefully considered before using it in a production environment, unless there is a patch in the repository for a particular show-\/stopping issue you are experiencing. It is generally a better idea to use a tagged release for stability.

\subsection*{License}

See \mbox{[}L\+I\+C\+E\+N\+SE\mbox{]}(L\+I\+C\+E\+N\+SE).

\subsection*{Installing the E\+V\+EN Platform A\+PI from a package}

Packages are available for


\begin{DoxyItemize}
\item Ubuntu 18.\+04 x64 L\+TS Linux\+: \href{https://evenfound.org/packages/event-network-ubuntu/}{\tt even-\/network-\/ubuntu}
\item Windows 10 M\+S\+Y\+S2 (mingw64)\+: \href{https://evenfound.org/packages/event-network-windows/}{\tt even-\/network-\/windows}
\end{DoxyItemize}

Packaging for your favorite distribution would be a welcome contribution!

\subsection*{Compiling the E\+V\+EN Platform Application from source}

\subsubsection*{On Linux\+:}

(Tested on Ubuntu 18.\+04 x64)


\begin{DoxyEnumerate}
\item Install E\+V\+EN Platform dependencies
\begin{DoxyItemize}
\item For Debian distributions (Ubuntu)

{\ttfamily sudo apt install build-\/essential cmake libboost-\/all-\/dev miniupnpc libunbound-\/dev graphviz doxygen libunwind8-\/dev pkg-\/config libssl-\/dev libzmq3-\/dev libsodium-\/dev libhidapi-\/dev}
\end{DoxyItemize}
\item Install Qt\+:

{\itshape Note}\+: Qt 5.\+10 is the minimum version required to build. This makes {\bfseries some} distributions (mostly based on debian, like Ubuntu 16.\+x or Linux Mint 18.\+x) obsolete. You can still build the G\+UI if you install an \href{https://wiki.qt.io/Install_Qt_5_on_Ubuntu}{\tt official Qt release}, but this is not officially supported.
\begin{DoxyItemize}
\item For Ubuntu 17.\+10+

{\ttfamily sudo apt install qtbase5-\/dev qtbase5-\/private-\/dev qt5-\/default qtdeclarative5-\/dev qml-\/module-\/qtquick-\/controls qml-\/module-\/qtquick-\/controls2 qml-\/module-\/qtquick-\/dialogs qml-\/module-\/qtquick-\/xmllistmodel qml-\/module-\/qt-\/labs-\/settings qml-\/module-\/qt-\/labs-\/folderlistmodel qttools5-\/dev-\/tools libqt5websockets5-\/dev qml-\/module-\/qtquick-\/templates2}
\end{DoxyItemize}
\item Clone repository and submodules to root of home directory

\`{}\`{}\`{} git clone \href{https://github.com/evenfound/even-network.git}{\tt https\+://github.\+com/evenfound/even-\/network.\+git} git submodule init git submodule update \`{}\`{}\`{}
\item Install cmake

{\ttfamily sudo apt-\/get install cmake}
\item Build

\`{}\`{}\`{} cd even-\/network Q\+T\+\_\+\+S\+E\+L\+E\+CT=5 ./build.ubuntu.\+sh \`{}\`{}\`{}
\end{DoxyEnumerate}

The executable can be found in the ./bin folder.

\subsubsection*{On Windows\+:}

The E\+V\+EN Platform on Windows is 64 bits only; 32-\/bit Windows builds are not officially supported anymore.


\begin{DoxyEnumerate}
\item Install \href{https://www.msys2.org/}{\tt M\+S\+Y\+S2}, follow the instructions on that page on how to update system and packages to the latest versions
\item Open an 64-\/bit M\+S\+Y\+S2 shell\+: Use the {\itshape M\+S\+Y\+S2 Min\+GW 64-\/bit} shortcut, or use the {\ttfamily msys2\+\_\+shell.\+cmd} batch file with a {\ttfamily -\/mingw64} parameter
\item Install M\+S\+Y\+S2 packages for E\+V\+EN Platform dependencies; the needed 64-\/bit packages have {\ttfamily x86\+\_\+64} in their names

\`{}\`{}\`{} pacman -\/S mingw-\/w64-\/x86\+\_\+64-\/toolchain make mingw-\/w64-\/x86\+\_\+64-\/cmake mingw-\/w64-\/x86\+\_\+64-\/boost mingw-\/w64-\/x86\+\_\+64-\/openssl mingw-\/w64-\/x86\+\_\+64-\/zeromq mingw-\/w64-\/x86\+\_\+64-\/libsodium mingw-\/w64-\/x86\+\_\+64-\/hidapi \`{}\`{}\`{}

You find more details about those dependencies in the \href{https://github.com/evenfound/even-network/doc}{\tt E\+V\+EN Platform documentation}. Note that that there is no more need to compile Boost from source; like everything else, you can install it now with a M\+S\+Y\+S2 package.
\item Install Qt5

\`{}\`{}\`{} pacman -\/S mingw-\/w64-\/x86\+\_\+64-\/qt5 \`{}\`{}\`{}

There is no more need to download some special installer from the Qt website, the standard M\+S\+Y\+S2 package for Qt will do in almost all circumstances.
\item Install git

\`{}\`{}\`{} pacman -\/S git \`{}\`{}\`{}
\item Clone repository and submodules to root of home directory

\`{}\`{}\`{} git clone \href{https://github.com/evenfound/even-network.git}{\tt https\+://github.\+com/evenfound/even-\/network.\+git} git submodule init git submodule update

\`{}\`{}\`{}
\item Build

\`{}\`{}\`{} cd even-\/network ./build.msys2.\+sh \`{}\`{}\`{}
\end{DoxyEnumerate}

The executable can be found in the {\ttfamily ./bin} directory. 