I\+P\+FS implementation in C, (not just an A\+PI client library).

\subsection*{Quick start for users\+:}


\begin{DoxyItemize}
\item {\bfseries ipfs init} to create an ipfs repository on your machine
\item {\bfseries ipfs add My\+File.\+txt} to add a file to the repository, will return with a hash that can be used to retrieve the file.
\item {\bfseries ipfs cat {\itshape hash}} to retrieve a file from the repository
\end{DoxyItemize}

\subsection*{For techies (ipfs spec docs)\+:}


\begin{DoxyItemize}
\item https\+://github.com/ipfs/specs/blob/master/overviews/implement-\/ipfs.\+md \char`\"{}getting started\char`\"{}
\item \href{https://github.com/ipfs/specs}{\tt specifications}
\item \href{https://github.com/ipfs/community/issues/177}{\tt getting started}
\item \href{https://github.com/libp2p/specs}{\tt libp2p}
\end{DoxyItemize}

\subsection*{Prerequisites\+: To compile the C version you will need\+:}


\begin{DoxyItemize}
\item \href{https://github.com/jmjatlanta/lmdb}{\tt lmdb}
\item \href{https://github.com/Agorise/c-protobuf}{\tt c-\/protobuf}
\item \href{https://github.com/Agorise/c-multihash}{\tt c-\/multihash}
\item \href{https://github.com/Agorise/c-multiaddr}{\tt c-\/multiaddr}
\item \href{https://github.com/Agorise/c-libp2p}{\tt c-\/libp2p}
\end{DoxyItemize}

And of course this project at \href{https://github.com/Agorise/c-ipfs}{\tt https\+://github.\+com/\+Agorise/c-\/ipfs}

The compilation at this point is simple, but not very flexible. Place all of these projects in a directory. Compile all (the order above is recommended) by going into each one and running \char`\"{}make all\char`\"{}. 